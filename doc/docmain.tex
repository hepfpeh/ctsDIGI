\documentclass{article}
\usepackage[letterpaper,margin=3cm]{geometry}
\usepackage[spanish,es-nodecimaldot,es-tabla]{babel}
\usepackage[utf8]{inputenc}

\usepackage{hyperref}
\usepackage{lipsum}
\usepackage{enumitem}
\setlist[description]{leftmargin=\parindent,labelindent=\parindent}

\title{Clase \texttt{ctsDIGI} para enviar y previsualizar artículos a la revista Ciencia, Tecnología y Salud de la DIGI}
\author{L.García, H. Pérez\footnote{\href{mailto:hector@ecfm.usac.edu.gt}{hector@ecfm.usac.edu.gt}}\\\small{\href{http://ecfm.usac.edu.gt}{Escuela de Ciencias Físicas y Matemáticas}, USAC}}
\date{Octubre 2016}

\begin{document}

\maketitle

%\begin{abstract}
%La clase \texttt{ctsDIGI} es una modificación del paquete \texttt{apa6}, por lo que están activias la mayoría de sus opciones. Al llamar el paquete con la opción \texttt{submit} se genera un documento con el formato que la DIGI exige para recepción de los artículos. La opción \texttt{preview} genera un documento que se parece bastante a la forma en que son publicados los artículos en la revista.
%\end{abstract}


\section{Introducción}
La idea y motivación de esta clase es facilitar el envío de artículos a la revista de \href{http://digi.usac.edu.gt/ojsrevistas/index.php/cytes}{Ciencia, Tecnología y Salud} de la \href{http://digi.usac.edu.gt/}{DIGI} para usuarios de \href{https://www.latex-project.org/}{\LaTeX}. 
El objetivo principal es proporcionar el formato para el envío de artículos, sin embargo la salida por defecto  emula la forma en que se verá el artículo en la revista. Esto es así para que los autores tengan una clara idea de como se verá su publicación y puedan hacer ajustes a figuras, tablas, ecuaciones, etc.

La clase \texttt{ctsDIGI} se basa en el paquete \href{https://www.ctan.org/pkg/apa6}{\texttt{apa6}}, por lo que están activias aun la mayoría de sus opciones. Los modificaciones hechas y los problemas aun no resueltos se detallan en las siguientes secciones.

\section{Contenido del paquete}
La versión mas reciente de la clase se obtiene del repositorio \href{https://github.com/hepfpeh/ctsDIGI.git}{https://github.com/hepfpeh/ctsDIGI.git} cuyo contenido es:
\begin{verbatim}
- guide.pdf            Este documento.
- README.md            Archivo Readme.
- doc                  [carpeta]
  - docmain.tex        Código fuente de este documento.
  - Makefile           Archivo de instrucciones para el comando make
- template             [carpeta]
  - bibliografia.bib   Base de datos bibliográfica de ejemplo.
  - ctsDIGI.cls        Archivo de definición de clase.
  - figura.pdf         Figura de ejemplo en pdf.           
  - main.tex           Archivo .tex de ejemplo de uso de la clase.
  - Makefile           Archivo de instrucciones para el comando make
\end{verbatim}

En la carpeta \texttt{template} se incluyen todos los archivos necesarios para generar un documento que utiliza la clase. Este no solo es un ejemplo sino que también puede ser utilizado como plantilla para otros documentos. Es necesario recordar que de momento esta clase \emph{no} se instala en el sistema, por lo que el archivo \texttt{ctsDIGI.cls} \emph{debe} estar en la misma carpeta que el \texttt{.tex} que la utiliza. La forma de compilar por primera vez es:
\begin{verbatim}
pdflatex main.tex
biber main.bcf
pdflatex main.tex
pdflatex main.tex
\end{verbatim}
También se incluye un \texttt{Makefile} en la carpeta, de forma que solo es necesario escribir \texttt{make} para compilar el documento y \texttt{make clean} para eliminar todos los archivos intermedios generados (conserva el \texttt{.pdf} de salida).

\section{La Clase}
La clase genera dos tipos de salida, que se escogen por medio de opciones: \texttt{submit} y \texttt{preview} que generan el formato apropiado para enviar artículos y previsualizar como se verán en línea o impresos en la revista.

Por el momento se ha implementado la clase modificando el archivo \texttt{apa6.cls} y renombrándolo como \texttt{ctsDIGI.cls}, por lo que la documentación de las opciones para el paquete \texttt{apa6} sigue siendo válida para la nueva clase.

\subsection{Comandos}
Los siguientes comandos se definieron en la clase:
\begin{description}
\item [\texttt{\textbackslash titleSp\{\}}] Título en español.
\item [\texttt{\textbackslash titleEn\{\}}] Título en inglés.
\item [\texttt{\textbackslash shorttitle\{\}}] Título corto.
\item [\texttt{\textbackslash author\{\}}] Autor(es).\footnote{Ver la documentación del paquete \href{https://www.ctan.org/pkg/apa6}{\texttt{apa6}} para la forma de incluir varios autores con distintas afiliaciones.}
\item [\texttt{\textbackslash affiliation\{\}}] Afiliación del/los autor(es).\footnotemark[\value{footnote}]
\item [\texttt{\textbackslash contactmail\{\}}] Dirección de e-mail de alguno de los autores.
\item [\texttt{\textbackslash abstractSp\{\}}] Resumen (español).
\item [\texttt{\textbackslash abstractEn\{\}}] Abstract (inglés).
\item [\texttt{\textbackslash keywordsSp\{\}}] Palabras clave (español).
\item [\texttt{\textbackslash keywordsEn\{\}}] Keywords (inglés).
\item [\texttt{\textbackslash ctsDIGIprintbibliography}] Genera la sección de referencias y finaliza la numeración de líneas. Este comando debe ir al final del documento.
\end{description}
Con excepción del comando \texttt{\textbackslash ctsDIGIprintbibliography} todos los demas van al inicio del documento antes del \texttt{\textbackslash maketitle}


\subsection{Formato de vista preliminar}
La vista preliminar del artúculo online e impreso se activa por medio de la opción \texttt{preview}:
\begin{verbatim}
\documentclass[preview]{ctsDIGI}
\end{verbatim}

Esta también es la opción por defecto por lo que no es necesario especificarla. Se recomienda hacer todos los ajustes de tamaño de gráficas, tablas y ecuaciones dentro de esta opción, ya que está planeado que esta sea la que se importe a la revista al momento de la edición.

Esta opción se modificó para parecerse lo más posible a la versión online de los artículos publicados. Hay varios campos de fechas que no se pueden modificar por el usuario, como la fecha de recepción, revisiones, publicación, etc.

Se debe tomar en cuenta que el aspecto final del artículo será decidido por los editores de la revista y no coincidirá al $100\%$ con este formato.

\subsection{Formato para envío de artículos}
Para habilitar esta salida se pasa la opción \texttt{submit} al momento de llamar la clase:
\begin{verbatim}
\documentclass[submit]{ctsDIGI}
\end{verbatim}

La clase \texttt{apa6} con la opción \texttt{man} ya genera un formato con la mayoría de requerimientos exigidos por la DIGI para la recepción de artículos, como los siguientes:
\begin{itemize}
\item Titlepage con título corto, título, autores con sus afiliaciones.
\item Página con abstract.
\item Texto con separación a doble línea.
\item Manda las figuras y tablas (floats) automáticamente al final del documento y las coloca en páginas individuales.
\item La bibliografía se coloca en el formato APA.
\end{itemize}

Las principales modificaciones respecto del formato \texttt{apa6} son las siguientes:
\begin{itemize}
\item Se incluyen los comandos para generar la dirección de contacto del autor, título en ingles y español, palabras clave en ingles y español, abstract en ingles y español.
\item Se numeran las líneas en el margen izquierdo a partir del abstract (resumen) y se finalizan antes de la bibliografía.
\item En las citas bibliográficas dentro del texto con multiples autores aparece \& en lugar de \emph{y}.
\end{itemize}

\subsection{Ecuaciones y \texttt{amsmath}}
Cuando se utiliza la opción \texttt{submit} para la salida existe un problema de conflicto entre los paquetes \href{https://www.ctan.org/pkg/lineno}{\texttt{lineno}} y\href{https://www.ctan.org/pkg/amsmath}{\texttt{amsmath}} ya que ambos redefinen los entornos de ecuaciones. Esto hace que los párrafos que contienen ecuaciones no sean numerados. Para resolverlo se incluye en la clase la opción \texttt{amsmath} que llama a los paquetes en el orden adecuado para solventar el conflicto. La forma de incluir a \texttt{amsmath} será:
\begin{verbatim}
\documentclass[...,amsmath,...]{ctsDIGI}.
\end{verbatim}
y no mediante \texttt{\textbackslash usepackage}.

También se debe observar que las ecuaciones \emph{no generan números de línea}, así que es mejor numerarlas todas para que al momento de la revisión se tenga un número de referencia. Este problema solo se presente en la opción \texttt{submit}.

Se recomienda ajustar los cambios de línea de ecuaciones largas con dentro de la la opción \texttt{preview}.

\subsection{Figuras y tablas (floats)}
Los entornos \texttt{figure} y \texttt{table} se comportan diferente según la opción de salida que se escoja. Con la opción \texttt{preview} se comportan de la forma habitual en \LaTeX, colocando las figuras y las tablas dentro del texto. Con la opción \texttt{submit} son enviadas automáticamente al final del documento y en páginas individuales.

El tipo de fuente de las descripciones (\emph{caption}) es fijado por la clase. Se recomienda ajustar el tamaño y ubicación de figuras y tablas con dentro de la la opción \texttt{preview}.

Dentro del documento de ejemplo en \texttt{template} se incluyen una figura y una tabla que se ajustan dentro una columna cuando el formato es a dos columnas. 

\subsubsection{Figuras}
Se recomienda que las figuras a incluir tengan el texto \emph{renderizado} ya dentro de ellas y que este proceso no se haga al momento de compilación, ya que el tamaño y tipo de fuente cambia según la opción de salida, lo cual puede ocasionar que la figura no se vea igual al cambiar la salida. El archivo \texttt{figura.pdf} es un ejemplo de figura con texto renderizado generada con \href{https://inkscape.org}{\texttt{inkscape}}. Las Figuras renderizadas al momento de compilación son aquellas que utilizan comandos para colocar el texto, como las que utilizan \href{http://tug.org/PSTricks/main.cgi/}{\texttt{pstricks}} o similares.

Es importante hacer notar que en las figuras, la descripción debe aparecer \emph{después} de las figuras. (ver el documento de ejemplo en \texttt{template})

\subsubsection{Tablas}
Hacer tablas en \LaTeX siempre ha sido un arte que requiere paciencia y esfuerzo. En el \texttt{.tex} de ejemplo en la carpeta \texttt{template} se muestra una tabla con el formato esperado de la revista. Algunos lineamientos para hacerlas son:
\begin{itemize}
\item La descripción va \emph{antes} de la tabla sin punto al final.
\item Tomar en cuenta las líneas horizontales que separan el cuerpo de la tabla de los encabezados.
\item Las notas al pie de la tabla deben contener la palabra \emph{Nota} en cursiva seguida del símbolo respectivo.
\end{itemize}



\subsection{Citas bibliográficas}
Las citas bibliográficas son manejadas por \texttt{biber} y obedecen al formato de \texttt{biblatex}, el cual acepta también el de \texttt{bibtex}. La diferencia principal es que \texttt{biber} tiene soporte para unicode. Véase que el archivo de ejemplo \texttt{bibliobrafia.bib} fue escrito con unicode. Para que las citas bibliográficas se ajusten al formato requerido es necesario incluir la opción \texttt{biblatex} al momento de llamar la clase
\begin{verbatim}
\documentclass[..., bibllatex,...]{ctsDIGI}.
\end{verbatim}
LLamar a \texttt{biblatex} por medio de \texttt{\textbackslash usepackage} no generará correctamente las citas bibliográficas dentro del texto para multiples autores.


\section{Notas}
\begin{itemize}

\item \texttt{biber} da problema con la forma tradicional de poner acentos en \TeX. Por Ejemplo: si se coloca en el archivo \texttt{.bib} la i con acento como\texttt{\textbackslash'\{\textbackslash i\} } esto dará un error de unicode. Hay que colocar \texttt{í}.

\item En la salida de \texttt{preview} el caracter (*) que se coloca en el nombre del autor para recibir correspondencia aparece también en el encabezado de todas las páginas.

\end{itemize}

\section{TODO}
\begin{enumerate}
\item Colocar tablas y figuras de ejemplo que se expandan a 2 columnas.
\item Probar la clase en combinación con otros paquetes y ver si no da algún conflicto.
\item Probar con los encargados de la editorial de la DIGI si pueden importar el texto a su programa.
\end{enumerate}




 


\end{document}
